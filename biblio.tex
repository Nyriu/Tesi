%!TEX TS-program = pdflatex
%!TEX root = tesi.tex
%!TEX encoding = UTF-8 Unicode

%https://www.unicusano.it/blog/didattica/corsi/sitografia-tesi/

\begin{thebibliography}{3}
\selectlanguage{english}
\frenchspacing

%\bibitem{}
%\emph{}.
%URL: \url{}.


\bibitem{mnist}
Y. LeCun, C. Cortes and C.J.C. Burges,
\emph{The MNIST Database of handwritten digits}.
URL: \url{http://yann.lecun.com/exdb/mnist/}.


\bibitem{cifar}
\emph{The CIFAR-10 dataset}.
URL: \url{https://www.cs.toronto.edu/~kriz/cifar.html}.

\bibitem{imagenet}
\emph{ImageNet}.
URL: \url{http://www.image-net.org/}.


\bibitem{resnet}
K. He, X. Zhang, S. Ren and J. Sun,
\emph{Deep Residual Learning for Image Recognition}.
(2015).
%https://arxiv.org/pdf/1512.03385.pdf
%@article{He2015DeepRL,
%  title={Deep Residual Learning for Image Recognition},
%  author={Kaiming He and Xiangyu Zhang and Shaoqing Ren and Jian Sun},
%  journal={2016 IEEE Conference on Computer Vision and Pattern Recognition (CVPR)},
%  year={2015},
%  pages={770-778}
%}


\bibitem{vgg}
K. Simonyan and A. Zisserman
, \emph{Very Deep Convolutional Networks for Large-Scale Image Recognition}.
(2015).
URL: \url{https://neurohive.io/en/popular-networks/vgg16/%https://neurohive.io/en/popular-networks/vgg16/}.


\bibitem{opencv}
\emph{OpenCV Documentation}.
URL: \url{https://docs.opencv.org/2.4/modules/imgproc/doc/imgproc.html}.


\bibitem{kernel-conv}
\emph{Kernel (image preprocessing)}.
URL: \url{https://en.wikipedia.org/wiki/Kernel_(image_processing)}.


\bibitem{deepai-feat-ext}
DeepAI,
\emph{What is Feature Extraction}.
URL: \url{https://deepai.org/machine-learning-glossary-and-terms/feature-extraction}.


\bibitem{ccir601}
\emph{CCIR 601 Standard}.
URL: \url{https://en.wikipedia.org/wiki/Rec._601}.

\bibitem{linear-interpolation}
Wikipedia,
\emph{Linear Interpolation}.
URL: \url{https://en.wikipedia.org/wiki/Linear_interpolation}.


\bibitem{bilinear-interpolation}
Wikipedia,
\emph{Bilinear Interpolation}.
URL: \url{https://en.wikipedia.org/wiki/Bilinear_interpolation}.

\bibitem{wikipedia-hist-eq}
Wikipedia,
\emph{Histogram Equalization}.
URL: \url{https://en.wikipedia.org/wiki/Histogram_equalization}.


\bibitem{hist-eq}
% TODO cite University Of California - Irvine ?
\emph{Histogram Equalization}.
URL: \url{https://www.math.uci.edu/icamp/courses/math77c/demos/hist_eq.pdf}.

\bibitem{wikipedia-gaussian-blur}
Wikipedia,
\emph{Gaussian Blur}.
URL: \url{https://en.wikipedia.org/wiki/Gaussian_blur}.


%% TODO DA SISTEMARE
%\bibitem{conv-ae}
%M. Comin,
%\emph{Convolutional Auto-Encoder}.
%URL: \url{https://ift6266mcomin.wordpress.com/2017/04/27/convolutional-auto-encoder-2/}.

%\bibitem{linear-classification}
%\emph{Linear Classification}.
%URL: \url{http://cs231n.github.io/linear-classify/}.

%\bibitem{fabric}
%\emph{Automatic Fabric Defect Detection with a Multi-Scale Convolutional Denoising Autoencoder Network Model}.
%URL: \url{https://www.ncbi.nlm.nih.gov/pmc/articles/PMC5948749/}.


\bibitem{bisker1}
J. Bi\v{s}ker, \emph{On the elements
of the empty set}. Mathematica Absurdica
\textbf{132} (1999), 13--113.


\bibitem{pyrl}
U. P\^{y}r{\l}\aa, \emph{Generalization
of Bi\v{s}ker's theorem}. Paperopolis
J. Math. \textbf{14} (2001), 125--132.


\end{thebibliography}
\selectlanguage{italian}
\nonfrenchspacing
