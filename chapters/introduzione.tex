%!TEX TS-program = pdflatex
%!TEX root = tesi.tex
%!TEX encoding = UTF-8 Unicode

\chapter{Introduzione}
\todo[inline]
{
  il probelma era tot
  l'obbiettivo era ottenere tot
  si è provato tot e tot
  risulta che gli autoencoder vanno bene
  cos'è un autoencoder
  come vengono usati di solito
}

\todo[inline]{TODO}
%In campo automotive, soprattutto negli ultimi anni, si è visto un crescente interesse nei confronti di sistemi di intelligenza artificiale che permettano di supervisionare la qualità dei pezzi prodotti.
%Assicurare la qualità è un requisito critico perché, in generale, la qualità influenza l'intera vita di un prodotto.
%Le applicazioni di algoritmi di Machine Learning prima, e di sistemi di Deep Learning poi, si sono dimostrate efficaci, flessibili e resilienti, portando numerosi vantaggi non solo nel campo del controllo automatico ma anche in quello del supporto agli operatori umani, ad esempio.
%I riferimenti alle applicazioni ben riuscite di sistemi intelligenti crescono di mese in mese ed offrono un ottimo mercato
%I riscontri 
%In un mercato 
%
%
%\todo[inline]{Spiegare cosa si intende con machine vision}
%
%\todo[inline]{Differenze machine vision vs. Computer vision}
%

% Machine vision
% % https://en.wikipedia.org/wiki/Machine_vision
% Computer vision
% % https://en.wikipedia.org/wiki/Computer_vision


% dire che librerie sono state usate
% dire che per noi True Positive vuole dire Scarto Scarto


\todo[inline]{START copiato da Autoencoder}

\clearpage
\section{Applicazioni principali}
Gli autoencoder non sono stati usati solamente come tecniche alternative per la \textit{dimensionality reduction}, seguono due delle principali applicazioni.

\paragraph{Denoising}
Con \textit{denoising} si intende rimuovere del rumore dai dati in \textit{input}.
All'AE viene richiesto, a partire da dati con del rumore


%\paragraph{Variational}

\paragraph{Anomaly Detection}


\todo[inline]{La nostra applicazione si trova a cavallo tra le due appena descritte}
\clearpage

\todo[inline]{END copiato da Autoencoder}
\clearpage
