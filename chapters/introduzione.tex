%!TEX TS-program = pdflatex
%!TEX root = tesi.tex
%!TEX encoding = UTF-8 Unicode

\chapter{Introduzione}

In campo automotive, soprattutto negli ultimi anni, si è visto un crescente interesse nei confronti di sistemi di intelligenza artificiale che permettano di supervisionare la qualità dei pezzi prodotti.
Assicurare la qualità è un requisito critico perché, in generale, la qualità influenza l'intera vita di un prodotto.
Le applicazioni di algoritmi di Machine Learning prima, e di sistemi di Deep Learning poi, si sono dimostrate efficaci, flessibili e resilienti, portando numerosi vantaggi non solo nel campo del controllo automatico ma anche in quello del supporto agli operatori umani, ad esempio.
I riferimenti alle applicazioni ben riuscite di sistemi intelligenti crescono di mese in mese ed offrono un ottimo mercato
I riscontri 
In un mercato 

TODO Sistemare sopra

TODO Spiegare cosa si intende con machine vision

Ref a

Machine vision
% https://en.wikipedia.org/wiki/Machine_vision
Computer vision
% https://en.wikipedia.org/wiki/Computer_vision
