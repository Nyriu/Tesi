%!TEX TS-program = pdflatex
%!TEX root = tesi.tex
%!TEX encoding = UTF-8 Unicode
\chapter{Abstract}
In questa tesi si affronta il problema di rilevare la presenza di colla sul
fondo di carcasse metalliche per motori elettrici.
Il \textit{dataset} fornito, composto da qualche migliaio di immagini, comprende due classi, ma una di queste è rappresentata da un numero veramente esigui di esemplari.
Si è deciso di utilizzare tecniche i \textit{Machine Learning} per il riconoscimento di anomalie.
In particolare è stato sviluppato un modello basato su reti neurali del tipo \textit{autoencoder}, poiché permettono di essere allenati utilizzando solamente una classe.
Applicando algoritmi di manipolazione e trasformazione delle immagini si è migliorato il \textit{dataset}.
La soluzione proposta si dimostra all'altezza dei risultati attesi, perché ha permesso di identificare quasi tutte le colle, senza però generare troppi falsi positivi.
In questo studio non è stato possibile determinare se la soluzione proposta sia applicabile in modo affidabile in ambienti industriali.
A tal fine sarebbe necessario ampliare il \textit{dataset} ed effettuare delle verifiche sul campo.
Il lavoro viene comunque considerato positivo perché ha permesso di esplorare le capacità combinate di algoritmi di manipolazione immagini e degli \textit{autoencoder}.


