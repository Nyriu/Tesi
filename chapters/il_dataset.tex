%!TEX TS-program = pdflatex
%!TEX root = tesi.tex
%!TEX encoding = UTF-8 Unicode

\chapter{Il Dataset}

Innanzitutto si specifica che con la parola Scarto si indica l'immagine di una carcasse che presenta colla sul fondo, pezzo che, quindi, dovrà essere scartato.
Invece con la parola Conforme si indica l'immagine di una carcassa nella quale la colla è stata depositata correttamente, quindi con fondo pulito.

Il dataset fornito e' fortemente sbilanciato, comprende 1749 immagini di cui:
\begin{itemize}
    \item 1719 sono Conformi;
    \item 30 sono Scarti.
\end{itemize}
I risultati illustrati in questo documento sono stati ottenuti utilizzando esclusivamente le immagini fornite o loro elaborazioni
Va fatto notare pero' che, qualora fosse stato necessario, si sarebbero potuti ottenere un numero arbitrariamente grande di Conformi, mentre il numero di Scarti sarebbe rimasto comunque esiguo.
%, inoltre gli Scarti forniti, per la natura della colla, non possono ritenersi esaustivi

TODO
Dimensioni immagini
qualità generali
 - cerchi concentrici
 - colori omogenei
 - la colla ha texture differente
 - la colla e' localizzata
 - Scarti sono asimmetrici




\cleardoublepage
%\section{Metriche di Valutazione} % TODO ??


\section{Problematiche Principali}
% Fortemente sbilanciato, falsi colori, contrasti differenti, centramenti
Osservando le immagini in figura, ci si accorge che le foto acquisite possono differire per vari motivi; verranno ora elencati in ordine di variabilita' partendo dalle proprieta' certe.
% TODO 3 immagini conformi vicine il piu' diverse possibile
 - La distanza dal fondo e' sempre quella (quindi le dimensioni relative degli anelli)
 - La superficie del pezzo varia ma bene o male e' sempre quello
 - centratura (varia ma sappiamo che non puo' esere completamente fuori)
 - luminosita' (ci possiamo basare sul piu' e sul meno luminoso)
 - riflessivita' del centro motlo fastidiosa
 - posizione delle balze (possono capitare a 360 gradi)

\section{Importanza Preprocessing}
% specificare librerie usate ??
% specificare bene che quelli che ora sono passaggi ben ordinati sono il risultato di mesi di trial and error
% Processo di modifica delle immagini ed euristiche applicate per ottenerne di "buone"

Si illustra il Preprocessing che ha portato ai migliori risultati

- centramento fondametale per aiutare la rete
    - decrizione funzione di centramento (passaggio BW, negativo, houghCirc media e matrice affine)
- passaggio da RGB a Grayscale perche non c'e' una grande perdita di informazione
- bilateralFilter per smoothare via la texture "sale e pepe" tipica delle carcasse
- il masking per rimovere le balze e l'area corrispondente alla zona verticale della carcassa
- crop per levare l'area nera (in proporzione ho molta piu' informazione)
- resize ad 224x224 perche' si e' dimostrata essere una buona dimensione (magari mostrare come sotto il 200 e' difficile vedere la colla, quelle piccole spariscono)

\subsection {Centramento mediante Hough Circles}
\subsection {Passaggio da RGB a GrayScale}
\subsection {Equalizzazione}
\subsection {Bilateral Filtering}
\subsection {Masking}
\subsection {Crop e Resizing}







\section {Data Augmentation}
% Com'è stato sfruttato la Data Aug
% Colle sintetiche
Due metodi principali:
 - rotazione
 - generazione degli scarti sintetici

\subsection {Rotazione}
Rotazione è metodo standard e largamente usato e bla bla

\subsection {Generazione Scarti Sintetici}


Scarti Sintetici
 I ritagli non sono stati semplicemente incollati sui conformi:

 la luminosita' della colla e' stata modificata per avvicinarsi a quella del pezzo conforme;
 dopo aver aggiunto la colla e' stato praticato uno smooth lungo il contorno, per evitare che ci fosse una transizione netta fra sofndo ed inizio bordo della colla.







































% TODO in caso sarebe un Chapter a se'
%\section {La strada proposta}
% Simile a quanto detto in considerazioni_pixelwise_diff
%Mostrare qual'e' l'obbiettivo che si vuole raggiungere con gli AE
%Spiegare che sono elastici e facili da allenare
%Il dataset non richiede dispendioso labeling
