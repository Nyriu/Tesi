%!TEX TS-program = pdflatex
%!TEX root = tesi.tex
%!TEX encoding = UTF-8 Unicode

\chapter{Il Dataset}

Innanzitutto si specifica che con la parola Scarto si indica l'immagine di una carcasse che presenta colla sul fondo, pezzo che, quindi, dovrà essere scartato.
Invece con la parola Conforme si indica l'immagine di una carcassa nella quale la colla è stata depositata correttamente, quindi con fondo pulito.

TODO recap capitolo
In questo capitolo verrà descritto il dataset e quali.


\section{Problematiche Principali}

\subsection{Dataset Piccolo e Sbilanciato}
La prima difficoltà insorge ancora prima di ispezionare le immagini del dataset.
Infatti il dataset non solo comprende solamente 1719 immagini ma è anche fortemente sbilanciato:
\begin{itemize}
    \item 1719 immagini sono Conformi;
    \item 30 immagini sono Scarti.
\end{itemize}

A questo punto è corretto chiedersi se le quasi duemila immagini del dataset siano sufficienti ai nostri scopi.
Nel campo del \textit{Machine Learning}, ed ancora di più in quello del \textit{Deep Learning}, non è raro che il numero di elementi in un dataset sia dell'ordine delle decine di migliaia se non di quello delle centinaia di migliaia.
% TODO aggiungere refs ai dataset
Basti pensare ai dataset più famosi ed usati:
\begin{itemize}
  \item MNIST è un dataset molto famoso contenente $70000$ cifre disegnate a mano appartenenti a $10$ classi in totale.
    È alla stesso tempo sia il punto di partenza dei principianti, perché di facile manipolazione, sia il campo di prova degli esperti, sul quale vengono allenati nuovi modelli prima di passare a compiti più complessi;
  \item CIFAR-10 contiene TODO descrivere 
  \item ImageNet contiene TODO descrivere 
\end{itemize}
%TODO aggiungere immagini esempio??

Ciascuno di questi dataset è stato etichettato\footnote{in inglese \textit{labeled} da \textit{label}, etichetta} a mano.
Ciò significa che ad ogni immagine è stata assegnata, a mano, una classe di appartenenza.
Prendendo in esempio MNIST, se un'immagine raffigura la cifra $7$ allora sarà etichettata con il label \textit{seven} ed apparterrà alla classe di immagini in cui compare la cifra $7$.
Allo stesso modo le immagini di ImageNet hanno etichette come \textit{dog}, \textit{cat}, \textit{bird}, \textit{car}, \textit{bike}, ecc... %TODO forma corretta??
Il nostro dataset, come già illustrato, contiene due classi: Conforme e Scarto.

Sembrerebbe che possedere 2000 immagini appena renda impossibile applicare algoritmi di \textit{Machine Learning}.
In realtà osservando i Conformi, in figura TODO sono stati riportati alcuni esemplari significativi, ci si accorge che i pezzi sono molto simili.
Le differenze principali tra un'immagine e l'altra riguardano la posizione delle balze, la luminosità e le imperfezioni superficiali (graffi, macchie, \dots), ciascuna di queste differenze verrà analizzata nel dettaglio tra poco.
Quindi, come ci si poteva aspettare essendo pezzi creati meccanicamente, la loro distribuzione è nell'intorno del pezzo progettato. TODO dire meglio.
Concludiamo che il numero di Conformi a nostra disposizione è sufficiente

Purtroppo non possiamo dire lo stesso per gli Scarti.
Se già la statistica ci lascia sospettare che $30$ esemplari non possono ritenersi significativi, allora questo sospetto diventa certezza quando si analizzano le caratteristiche della colla nelle immagini Scarto.  
Come si vede in figura TODO
la colla può presentarsi in forma di gocce più o meno circolari oppure come sbaffi di grossezza e lunghezza variabili.
Anche la quantità di superficie coperta dalla colla può variare notevolmente, passando da aree ridotte e localizzate ad aree estese e di conformazioni singolari.
Infine notiamo che la posizione del rimasuglio di colla all'interno della carcassa non è in relazione con la posizione delle balze e che la presenza dei gradini sul fondo non la obbliga in alcun modo a scivolare fino al centro.

Per analizzare meglio le modalità con cui potrebbe essere generato uno Scarto si supponga che il macchinario abbia commesso un errore: dall'ugello è uscita un certa quantità di colla in esubero.
A seconda della posizione dell'ugello rispetto alla carcassa si può immaginare che la colla raggiunga il fondo in vari modi, proviamo ora ad esplorarne due:
\begin{itemize}
  \item nel primo caso si immagina che il braccio abbia già depositato l'anello di colla e che si stia allontanando dalla carcassa.
    La colla in esubero cadrebbe sotto forma si goccia fino a raggiungere il fondo del pezzo.
    Questo potrebbe esse il caso per la figura TODO ref immagine con colla a goccia;
  \item nel secondo caso si immagina che la colla in esubero faccia parte dell'anello appena depositato e che, a causa delle vibrazioni o di altri fattori simili, coli raggiungendo il fondo della carcassa.
    Questo potrebbe esse il caso per la figura TODO ref immagine con colla "sbaffata" dal bordo.
\end{itemize}

Concludiamo che gli esemplari forniti per la classe Scarto descrivono soltanto in modo parziale la distribuzione della classe (TODO dire meglio) e che quindi non possono essere utilizzati per allenare un modello veramente generale.
Infatti se, per esempio, venissero usati per il train di una rete convolutiva, di cui poi verrà illustrata brevemente la struttura, si rischierebbe di creare un modello con forte \textit{overfit} rispetto a quelle specifiche macchie di colla fornite.
Con il termine overfit si intende TODO.
%Dopo queste osservazioni siamo convinti che il problema deve essere affrontato com e un problema di anomaly detection... Piccola introduzione? Meglio parlarne vicino agli AE


Prima di proseguire con le prossime problematiche dobbiamo spendere alcune parole per commentare la colorazione delle immagini rispetto ai veri colori delle carcasse e della colla.
In figura~\ref{fig:carc} a pagina \pageref{fig:carc} abbiamo visto che la superficie del pezzo è di colore grigio ma nella foto risulta di colore verdastro.
Allo stesso modo anche la colla, in realtà di colore bianco sporco, nella fotografia assume tonalità verdognole.
Per certi compiti possedere immagini in falsi colori può risultare problematico ma fortunatamente non è questo il caso: l'importante è che venga mantenuta l'informazione che ci permette di distinguere la colla dalla superficie della carcassa.
Come vedremo poi le immagini verranno trasformate in scala di grigi quindi, nonostante sarebbe stato preferibile avere immagini a colori reali, i falsi colori non sono da considerarsi problematici.

\subsection{Differenze tra Immagini}
Ora che abbiamo una visione d'insieme sul dataset possiamo concentrare la nostra attenzione sulle proprietà principali delle immagini.
%TODO verificare dimensione immagine
Innanzitutto ogni immagine ha una risoluzione di $896$x$896$ pixel, dimensione che ci permette di esplorare varie possibilità.
Ad esempio si può pensare di ridurre l'immagine ad una dimensione tale da: occupare meno spazio in memoria, quindi in RAM durante il training, ed allo stesso tempo di mantenere un livello di dettaglio sufficiente ai nostri scopi, risultando quindi in un boost in velocità di train. TODO dire meglio.
Oppure di suddividere l'immagine in quadranti da analizzare singolarmente così da mantenere la qualità dell'immagine originale ma senza dover creare una rete che accetti immagini troppo grandi.
%TODO spiegare meglio
Infatti una rete che accetta immagini di grandi dimensioni, solitamente, avrà un numero di paramatri maggiore di una che accetta immagini piccole.
Questo porta non solo ad occupare più spazio in memoria ma significa anche che la rete impiegherà più tempo sia in fase di train (più parametri da aggiustare) sia in fase di predizione (più conti da fare). TODO sistemare.
%TODO verificare dimensione immagine
Per avere un termine di paragone basti pensare che le immagini di MNIST sono $64$x$64$ pixel mentre quelle di ImageNet di $TODO$x$TODOpx$.

%% falsi colori, contrasti differenti, centramenti
Osservando nuovamente Figura TODO ci si accorge che le immagini hanno varie proprietà, verranno ora elencate e commentate in ordine crescente di fastidiosità (TODO fix).
% le prime sono da considerarsi poco problematiche o non problematiche le ultime problematiche
\begin{itemize}

  \item Ogni immagine presenta tre circonferenze concentriche, con centro il centro del pezzo.
    Ciascuna circonferenza è definita da una transizione da una zona più scura ad una più chiara.
    Sappiamo che le zone più scure corrispondono alle pareti verticali del pezzo mentre le zone chiare ai tre gradini sul fondo.
    Questa proprietà non è problematica, anzi potrà essere sfruttata a nostro vantaggio.

  \item La distanza dal fondo è sempre quella (quindi le dimensioni relative degli anelli).

  \item Le due balze sulla parete verticale sono ben visibili e possono presentarsi, sempre una di fronte all'altra, in ogni posizione lungo una circonferenza di raggio pari al raggio della cavità cilindrica.
    Possono essere considerate un problema in quanto rappresentano informazione superflua e variabile.
    Ricordiamo che la posizione delle balze non ha alcuna correlazione con la presenza della colla, tanto meno con la sua posizione.

  \item La superficie del pezzo varia ma bene o male è sempre quello.
    Nota sulle macchie scure.

  \item centratura (varia ma sappiamo che non può essere completamente fuori)
  
  \item riflessività del centro fastidiosa

  \item luminosità (ci possiamo basare sul più e sul meno luminoso per avere più o meno il range) fastidiosa

\end{itemize}



% TODO 3 immagini conformi vicine il più diverse possibile
Ora possiamo elencare le proprietà esclusive degli Scarti, sempre in ordine di fastidiosità:
\begin{itemize}
  \item la colla ha texture differente

  \item la colla è localizzata

  \item Scarti sono asimmetrici

\end{itemize}

% pezzo sotto dirlo ora o quando faccio la diff la prima volta nella sezione degli AE? Oppure quando creo gli scarti sintetici?
% Come ultima cosa accennare al fatto che gli scarti presentano anelli di colorazione differente e che questo sarà un grave problema in fase di valutazione dei risultati
% Specificare che NON è una proprietà "corretta" ma che dipende dal modo in cui sono state raccolte le immagini
% Che gli scarti on-line saranno uguali a Conforme+Colla




\section{Importanza Preprocessing}
% specificare librerie usate ??
% specificare bene che quelli che ora sono passaggi ben ordinati sono il risultato di mesi di trial and error
% Processo di modifica delle immagini ed euristiche applicate per ottenerne di "buone"

%Si illustra il Preprocessing che ha portato ai migliori risultati
%
%- centramento fondamentale per aiutare la rete
%    - descrizione funzione di centramento (passaggio BW, negativo, houghCirc media e matrice affine)
%- passaggio da RGB a Grayscale perché non c'è una grande perdita di informazione
%- bilateralFilter per smoothare via la texture "sale e pepe" tipica delle carcasse
%- il masking per rimuovere le balze e l'area corrispondente alla zona verticale della carcassa
%- crop per levare l'area nera (in proporzione ho molta più informazione)
%- resize ad 224x224 perché si è dimostrata essere una buona dimensione (magari mostrare come sotto il 200 è difficile vedere la colla, quelle piccole spariscono)

\subsection {Centramento mediante Hough Circles}
\subsection {Passaggio da RGB a GrayScale}
\subsection {Equalizzazione}
\subsection {Bilateral Filtering}
\subsection {Masking}
\subsection {Crop e Resizing}







\section {Data Augmentation}
% Com'è stato sfruttato la Data Aug
% Colle sintetiche
Due metodi principali:
 - rotazione
 - generazione degli scarti sintetici

\subsection {Rotazione}
Rotazione è metodo standard e largamente usato e bla bla

\subsection {Generazione Scarti Sintetici}


Scarti Sintetici
 I ritagli non sono stati semplicemente incollati sui conformi:

 la luminosità della colla è stata modificata per avvicinarsi a quella del pezzo conforme;
 dopo aver aggiunto la colla è stato praticato uno smooth lungo il contorno, per evitare che ci fosse una transizione netta fra sofndo ed inizio bordo della colla.







































% TODO in caso sarebe un Chapter a sé
%\section {La strada proposta}
% Simile a quanto detto in considerazioni_pixelwise_diff
%Mostrare qual'è l'obbiettivo che si vuole raggiungere con gli AE
%Spiegare che sono elastici e facili da allenare
%Il dataset non richiede dispendioso labeling
