%!TEX TS-program = pdflatex
%!TEX root = tesi.tex
%!TEX encoding = UTF-8 Unicode

\chapter{Metodi Alternativi Falliti}
% TODO da dire in introduzione

%Segue un breve elenco di metodi e tecniche alternative, nell'ordine in cui sono stati provati.


\section{ResNet e One Class SVM}
%http://rvlasveld.github.io/blog/2013/07/12/introduction-to-one-class-support-vector-machines/
Riconoscendo che il principale problema del \textit{dataset} è il fatto che sia sbilanciato, si è pensato a tecniche come la \textit{One Class Support Vector Machine}(OCSVM).
Questa è una variazione dell'algoritmo di classificazione \textit{Support Vector Machine} pensato per poter essere allenato avendo dati di un'unica classe.
Il modo più intuitivo per descrivere il funzionamento di una OCSVM è immaginare che il suo scopo sia creare la più piccola sfera contenente tutti i punti del \textit{dataset}.
Così facendo tutti gli elementi distanti dal \textit{dataset} vengono considerati anomalie perché non cadono all'interno della sfera.

Dato che l'informazione a nostra disposizione è contenuta in immagini si è dovuto deciso di usare una rete ResNet18~\cite{resnet} come \textit{feature extractor}








% ResNet + OCSVM 2 pagine

% tecniche classiche come quella roba della texture

% tecniche classiche come HOG

% AE solo fully connected - interessante vedere come l'immagine venisse imparata a pixel - 1pagina + qualche immagine?

% AE solo convolutivi - incapacità di astrazione - mezza pagina

% Symmetric Skip e come ricrea la colla 1 pagina

% immagini a patch

% logpolar immagini

%\section{Autoencoder e OCSVM}
