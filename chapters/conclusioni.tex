%!TEX TS-program = pdflatex
%!TEX root = tesi.tex
%!TEX encoding = UTF-8 Unicode

\chapter{Conclusioni}
Si ricorda che uno dei principali obbiettivi era quello di non scartare più del 2\% di esemplari conformi, riconoscendo il maggior numero possibile di scarti.
La soluzione proposta può essere divisa in tre momenti:
nel primo l'immagine in ingresso viene manipolata da algoritmi creati appositamente, così da ottenere un'immagine che favorisce le future trasformazioni;
l'immagine appena generata viene fornita in ingresso ad un \textit{autoencoder} con una specifica architettura;
l'\textit{output} della rete neurale viene gestito da un ulteriore algoritmo che permette di sintetizzare una classificazione binaria.
Osservando i risultati ottenuti si può vedere che, nonostante il campione a disposizione abbia una numerosità ridotta, nessun conforme dell'insieme di \textit{test} è stato classificato in modo scorretto.
Questo significa che c'è una quantità di falsi positivi, cioè di carcasse conformi che verrebbero scartate, pari allo 0\%.
Verificando le capacità del sistema anche sull'insieme d'allenamento, risulta che lo 1.1\% di elementi è stato classificato come scarto.
Tra questi si trovano principalmente foto scattate ad una distanza superiore alla media.
Il numero di scarti correttamente identificato corrisponde al 93.3\%.
Si può concludere che la soluzione proposta ha dato risultati vicini a quelli attesi e che quindi sia soddisfacente.
Il \textit{dataset} ristretto, però, non permette di determinare se la soluzione proposta sia applicabile in modo affidabile in ambienti industriali.
Studi futuri potrebbero comprendere un campione più ampio sia di conformi che di scarti e sperimentazioni con \textit{autoencoder} con architetture differenti.
